\documentclass[12pt]{article}

\newcommand{\project}[1]{\textsl{#1}}
\newcommand{\acronym}[1]{{\small{#1}}}
\newcommand{\LAMOST}{\project{\acronym{LAMOST}}}
\newcommand{\TheCannon}{\project{The~Cannon}}
\newcommand{\Gaia}{\project{Gaia}}

\begin{document}\raggedbottom\sloppy\sloppypar\frenchspacing

\paragraph{Abstract:}
Using Ting-built spectral derivatives,
we can assemble a derivative of stellar spectral expectation with respect
to a Earth-like composition anomaly.
We can then---by various data-driven means---locate stars in the
\LAMOST\ data that show evidence of having (recently?) consumed
large quantities of rock.

\biskip

For context for what follows, recall that Andy Casey (Monash) and Anna
Y. Q. Ho (Caltech) performed fits (with \TheCannon) to
\LAMOST\ spectra and searched the residuals for evidence of
Li-rich stars.
They could have performed this fit by projecting the
residals against a derivative of spectral expectation with respect
to Li abundance.
That is, you can find abundance anomalies by projecting residuals away
from a data-driven fit onto spectral derivatives.

For further context, Semyeong Oh (Princeton) has found a pair of similar
(and co-moving) G-type dwarf stars in \Gaia\ data, where one of the pair
shows an enormous abundance anomaly that is chemically similar to the
bulk composition of the Earth.
This anomaly is not subtle; it would have been easy to find with a matched
filter.

\end{document}
