\documentclass[12pt]{article}

\newcommand{\project}[1]{\textsl{#1}}
\newcommand{\acronym}[1]{{\small{#1}}}
\newcommand{\LAMOST}{\project{\acronym{LAMOST}}}
\newcommand{\TheCannon}{\project{The~Cannon}}
\newcommand{\Gaia}{\project{Gaia}}

\newcommand{\e}{\mathrm{e}}
\newcommand{\dd}{\mathrm{d}}
\newcommand{\teff}{T_{\mathrm{eff}}}
\newcommand{\logg}{\log g}
\newcommand{\feh}{[\mathrm{Fe}/\mathrm{H}]}
\newcommand{\rock}{\mathrm{rock}}

\begin{document}\raggedbottom\sloppy\sloppypar\frenchspacing

\paragraph{A data-driven approach to finding stars that eat rocks}
~

\noindent
by \textbf{Hogg} and others%
\footnote{Important contributions have already come from Ting, Brewer, and Ho.}

\paragraph{Abstract:}
Using Ting-built spectral derivatives,
we can assemble a derivative of stellar spectral expectation with respect
to a rocky composition anomaly.
We can then---by various data-driven means---locate stars in the
\LAMOST\ data that show evidence of having (recently?) consumed
large quantities of rock.

\paragraph{Introduction:}
For context for what follows, recall that Andy Casey (Monash) and Anna
Y. Q. Ho (Caltech) performed fits (with \TheCannon) to
\LAMOST\ spectra and searched the residuals for evidence of
Li-rich stars.
They could have performed this fit by projecting the
residuals against a derivative of spectral expectation with respect
to Li abundance.
That is, you can find abundance anomalies by projecting residuals away
from a data-driven fit onto spectral derivatives.

For further context, Semyeong Oh (Princeton) has found a pair of similar---in
terms of effective temperature, gravity, and composition---and
co-moving G-type dwarf stars in \Gaia\ data, where one of the pair
shows an enormous abundance anomaly that is chemically similar to the
bulk composition of the Earth.
This anomaly is not subtle; it would have been easy to find with a matched
filter.

For yet further context, Yuan-Sen Ting (Harvard) has computed derivatives
of the spectral expectation (the spectrum, really) for a star with parameters
$(\teff, \logg, \feh)$ with respect to element $k$ abundance ratio $\ln Q_k$,
where the abundance is taken relative to iron (or other reference element):
\begin{eqnarray}
\ln Q_k &\equiv& \frac{[X_k / \mathrm{Fe}]}{\log_{10} \e}
\quad ,
\end{eqnarray}
where $X_k$ is the absolute abundance of element $k$.

\paragraph{Method:}
Choose a small bin in spectral parameters $(\teff, \logg, \feh)$.
Compute or obtain (from Ting) spectral derivatives $\dd f_\lambda/\dd\ln Q_k$ for
all elements $k$ for the normalized spectra $f_\lambda$, relevant
to the stars in this small spectral-parameter bin.
That is, evaluate these derivatives at the parameters that are
appropriate for the stars in the bin.
(We always work in normalized-spectrum space in what follows; normalization
is presumed to be done, done correctly, and out of scope here.)

Obtain (from Oh) the derivatives $\dd\ln Q_k/\dd M_\rock$ which tell
you the logarithmic increase in the abundance of element $k$ per unit
mass of rock dumped into a star with parameters $(\teff, \logg, \feh)$
appropriate to this spectral-parameter bin.
These derivatives are just the rock abundances divided by the stellar
abundances, and maybe some factor that accounts for the photospheric
mass of the star with parameters $(\teff, \logg, \feh)$.
Take the dot product
\begin{eqnarray}
  \frac{\dd f_\lambda}{\dd M_\rock} &=&
  \sum_{k=1}^K \frac{\dd f_\lambda}{\dd\ln Q_k}\,\frac{\dd\ln Q_k}{\dd M_\rock}
  \quad .
\end{eqnarray}
This is the derivative of the spectral expectation with respect to
the amount of swallowed rock, relevant to stars in this bin.

Now make a \emph{test set} of
stars that have spectra consistent with being in or near
the bin centered on $(\teff, \logg, \feh)$.
This test-set membership ought to be made loose, especially in the $\feh$
direction, because we are looking for anomalies!

For each star $n$ in this test set, do the following:
$\bullet$~Obtain a spectral expectation for star $n$ (more below).
$\bullet$~Subtract the spectral expectation to create a residual spectrum
          $\Delta f_\lambda$ for star $n$.
$\bullet$~Project the residual spectrum onto the rock derivative
          $\dd f_\lambda / \dd M_\rock$.
$\bullet$~Threshold the projection and append to the list of stars with
          high values of the projection (or not).
(And then publish, get famous, win prizes, retire early.)

As for the spectral expectation mentioned above, there are a few ideas.
One bad idea is to use physical stellar models. These are bad, and they
are particularly bad where we care about our residuals. One good idea
is to use \TheCannon, trained on a good training set that overlaps the
spectral bin.
Another good idea is to just make an empirical difference between star
$n$ and stars that are (empirically, in the data) spectrally similar to
star $n$, but \emph{in the tangent space} of the derivative 
${\dd f_\lambda}/{\dd M_\rock}$.
That is, similar in all spectral directions \emph{orthogonal}
to the derivative we care about.

The key thing is that we want this spectral expectation to be data-driven,
or empirical.
Physical models just aren't good enough yet.

I got push-back from some on this latter statement, so let me just note
that we \emph{could} look for relevant stars in the chemical abundance
space directly.
That is, we could measure all the abundances, and look for stars that
are anomalous in the rock direction.
That is, after all, how we found Kronos!
I am happy for us to do that, but it has a number of disadvantages:
The first is that it depends more strongly on absolute calibration aspects
of the models.
The second is that, even if the models are good enough to make the correct
measurements, it is expected to be at least slightly less sensitive,
unless the search is done making responsible use of the full posterior
information about every star.
The third is that the search proposed here is very light-weight; it would
only take minutes to run on everything in
the \LAMOST\ data set.
That is, this search can be done without any long-term commitment to
(or any requirement to explain and execute) a full abundance analysis
of all stars.


\end{document}
